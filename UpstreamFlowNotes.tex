\documentclass[10pt,a4paper]{report}
\usepackage{minted}
\usepackage{listings}
\usepackage[paperheight=11in,paperwidth=9.5in,margin=1cm]{geometry}
\usepackage{multicol}
\usepackage{lscape}

\begin{document}
\begin{landscape}
\begin{multicols}{2}
%\lstset{numbers=left,numbersep=10pt,language=c++}
\begin{center}
\textbf{\underline{Notes for ACM ICPC by Upstream Flow(Bhavit,Prateek,Chirag)}}
\end{center}

	\underline{\textbf{Basic Template}}
\begin{minted}[frame=single,linenos=false,obeytabs=true,tabsize=1]{c++}
#define un(x) x.erase(unique(all(x)), x.end())

LL MOD = (LL)1e9 + 7; 
LL mulmod(LL a, LL b, const LL & m) {
	LL q = (LL)(((LD)a*(LD)b)/(LD)m); 
	LL r = a*b - q*m; 
	if (r>m)r %= m; 
	if (r<0)r += m; 
	return r; 
}

template <typename T, typename S>
T expo(T e, S n) { 
	T x = 1, p = e; 
	while (n){
		if (n & 1)x = x*p; p = p*p;
		n >>= 1; 
	}
	return x;
}

template <typename T>
T powerL(T e, T n, const T & m){
	T x = 1, p = e; 
	while (n){
		if (n & 1)x = mulmod(x, p, m); 
		p = mulmod(p, p, m); 
		n >>= 1; 
	}
	return x;
}
\end{minted}
\newpage
\begin{center}
	\underline{\textbf{String Algorithms}}
\end{center}

\begin{flushleft}
	\underline{KMP Algorithm}
\end{flushleft}

\begin{minted}[frame=single, autogobble,linenos=true, linenos=false]{c++}
vector<int> kmp(string & KMP){
    int KMPN = (int)KMP.size(); int q = -1;vector<int> pi(KMPN); ``
    pi[0] = -1;
    for(int i = 1; i < KMPN; i++){
        while(q >= 0 &&  KMP[i] != KMP[q + 1])
            q = pi[q];
        if(KMP[i] == KMP[q + 1])
            q++;
        pi[i] = q;
    }
    return pi; 
}
int Q_ = -1; 
for(int i = 0; i < n ;i++) {
    while(Q_ >= 0 && s[i] != t[Q_ + 1])
        Q_ = pi[Q_];
    if(s[i] == t[Q_ + 1])
        Q_++;
    if(Q_ == m - 1) {//string matched. 
        Q_ = pi[Q_];
    }
}
\end{minted}

\begin{flushleft}
\underline{Z Algorithm}
\end{flushleft}
\begin{minted}[frame=single, linenos=false, autogobble,linenos=true]{c++}

vector<int> z_compute(string & S_) {
	int zn = S_.size();
	vector<int> z((int)S_.size() + 10); 
	int l = 0, r = 0; z[0] = 0; 
	for (int i = 1; i < zn; i++) {
		if (i <= r)
			z[i] = min(r - i + 1, z[i - l]);
		while (i + z[i] < zn && S_[z[i]] == S_[i + z[i]])
			++z[i];
		if (i + z[i] - 1 > r)
			l = i, r = i + z[i] - 1;
	}
	return z; 
}
\end{minted}

\begin{flushleft}
\textbf{\underline{Suffix Array and LCP Array}}
\end{flushleft}

\begin{minted}[autogobble,linenos=true,frame=single,linenos=false,breaklines]{c++}

bool comp(pii a, pii b) {
	return (a.first.first == b.first.first ? a.first.second < b.first.second : a.first.first < b.first.first);
} int suffix[20][N]; vi suff,lcp,inv_suff;
void suffix_array() {
	for (int i = 0; i < n; i++)
		suffix[0][i] = s[i] - 'a';
	vector<pii> L; L.clear(); L.resize(n);
	for (int cnt = 1, stp = 1, cur_rank = 0; cnt < n; cnt <<= 1, cur_rank = 0, ++stp) {
		for (int i = 0; i < n; i++) {
			L[i].first.first = suffix[stp - 1][i];
			L[i].first.second = (i + cnt < n ? suffix[stp - 1][i + cnt] : -64);
			L[i].second = i;
		}

		sort(L.begin(), L.end(), comp);
		suffix[stp][L[0].second] = 0;

		for (int i = 1; i < n; i++) {
			if (L[i].first.first != L[i - 1].first.first || L[i].first.second != L[i - 1].first.second) {
				cur_rank++;
			}
			suffix[stp][L[i].second] = cur_rank;
		}
	}
	suff.clear(); inv_suff.clear(); inv_suff.resize(n);
	for (int i = 0; i < n; i++) {
		suff.push_back(L[i].second);
		inv_suff[L[i].second] = i;
	}
}
void LCP() {
	lcp.clear(); lcp.resize(n);
	for (int i = 0, l = 0; i < n; i++) {
		int j = inv_suff[i];
		if (j > 0) {
			int k = suff[j - 1];
			l = max(l - 1, 0);
			while (s[k + l] == s[i + l] && s[k+l] != '$')
				l++;
			lcp[j] = l;}}}
\end{minted}

\begin{center}
\textbf{\underline{Graph Algorithms}}
\end{center}

\begin{flushleft}
\textbf{\underline{Articulation Points and Bridges}}
\end{flushleft}

\begin{minted}[autogobble,linenos=true,frame=single,linenos=false,breaklines]{c++}
int dis[N], low[N], par[N], AP[N],vis[N],tits; 
void update(int u , int i, int child) {
    //For Cut Vertices
    if(par[u] != -1 && low[i] >= dis[u]) AP[u] = true;
    if(par[u] == -1 && child > 1) AP[u] = true;
    
    //For Finding Cut Bridge
    if(low[i] > dis[u]){
        //articulation bridge found.
    }
}
void dfs(int u){
    vis[u] = true;
    low[u] = dis[u] = (++tits); int child = 0;
    for(int i : g[u]) {
        if(!vis[i]){
            child++;
            par[i] = u;
            dfs(i);
            low[u] = min(low[u] , low[i]);
            update(u, i, child);
        }
        else if(i != par[u]) {
            low[u] = min(low[u] , dis[i]);
        }
    }
}
\end{minted}

\begin{flushleft}
\textbf{\underline{Union Find}}
\end{flushleft}

\begin{minted}[autogobble,linenos=true,frame=single,linenos=false,breaklines]{c++}
vi root, ra; //Set root[i]=i initially and all ranks to be zero 
int findset(int i) {
	return (root[i] == i) ? i : (root[i] = findset(root[i]));
}
bool isSameSet(int i, int j) {
	return findset(i) == findset(j);
}
void unionSet(int i, int j) { 
	if (!isSameSet(i, j)) {
		int x = findset(i), y = findset(j);
		if (ra[x] > ra[y]) { root[y] = x;  }
		else { 
			root[x] = y;
			if (ra[x] == ra[y]) ra[y]++; 
		}
	}
}
\end{minted}

\begin{flushleft}
\textbf{\underline{Max Flow Algorithm}}
\end{flushleft}

\begin{minted}[autogobble,linenos=true,frame=single,linenos=false,breaklines]{c++}
struct flow_graph{
    int MAX_V,E,s,t,head,tail;
    int *cap,*to,*next,*last,*dist,*q,*now;
    flow_graph(){}
    flow_graph(int V, int MAX_E){
        MAX_V = V; E = 0;
        cap = new int[2*MAX_E], to = new int[2*MAX_E], next = new int[2*MAX_E];
        last = new int[MAX_V], q = new int[MAX_V], dist = new int[MAX_V], now = new int[MAX_V];
        fill(last,last+MAX_V,-1);
    }
    void clear(){
        fill(last,last+MAX_V,-1);
        E = 0;
    }
    void add_edge(int u, int v, int uv, int vu = 0){
        to[E] = v, cap[E] = uv, next[E] = last[u]; last[u] = E++;
        to[E] = u, cap[E] = vu, next[E] = last[v]; last[v] = E++;
    }
	bool bfs(){
		fill(dist,dist+MAX_V,-1);
		head = tail = 0;
		
		q[tail] = t; ++tail;
		dist[t] = 0;
		
		while(head<tail){
			int v = q[head]; ++head;
			
			for(int e = last[v];e!=-1;e = next[e]){
				if(cap[e^1]>0 && dist[to[e]]==-1){
					q[tail] = to[e]; ++tail;
					dist[to[e]] = dist[v]+1;
				}
			}
		}
		return dist[s]!=-1;
	}
	int dfs(int v, int f){
		if(v==t) return f;
		
		for(int &e = now[v];e!=-1;e = next[e]){
			if(cap[e]>0 && dist[to[e]]==dist[v]-1){
				int ret = dfs(to[e],min(f,cap[e]));
				
				if(ret>0){
					cap[e] -= ret;
					cap[e^1] += ret;
					return ret;
				}
			}
		}
		return 0;
	}
	long long max_flow(int source, int sink){
		s = source; t = sink;
		long long f = 0;
		int x;
		
		while(bfs()){
			for(int i = 0;i<MAX_V;++i) now[i] = last[i];
			
			while(true){
				x = dfs(s,INT_MAX);
				if(x==0) break;
				f += x;
			}
		}
		return f;
	}
}G;
int main(){
	int V,E,u,v,c;
	scanf("%d %d",&V,&E);
	G = flow_graph(V,E);
	for(int i = 0;i<E;++i){
		scanf("%d %d %d",&u,&v,&c);
		G.add_edge(u-1,v-1,c,c);
	}
	printf("%lld\n",G.max_flow(0,V-1));
	return 0;
}
\end{minted}

\begin{flushleft}
\textbf{\underline{Undirected Maximum Flow}}
\end{flushleft}

\begin{flushleft}
\textbf{\underline{Maximal BiPartite Matching}}
\end{flushleft}

\begin{minted}[autogobble,linenos=true,frame=single,linenos=false,breaklines]{c++}
// This code performs maximum bipartite matching.
//
// Running time: O(|E| |V|) -- often much faster in practice
//
//   INPUT: w[i][j] = edge between row node i and column node j
//   OUTPUT: mr[i] = assignment for row node i, -1 if unassigned
//           mc[j] = assignment for column node j, -1 if unassigned
//           function returns number of matches made

#include <vector>

using namespace std;

typedef vector<int> VI;
typedef vector<VI> VVI;

bool FindMatch(int i, const VVI &w, VI &mr, VI &mc, VI &seen) {
  for (int j = 0; j < w[i].size(); j++) {
    if (w[i][j] && !seen[j]) {
      seen[j] = true;
      if (mc[j] < 0 || FindMatch(mc[j], w, mr, mc, seen)) {
        mr[i] = j;
        mc[j] = i;
        return true;
      }
    }
  }
  return false;
}

int BipartiteMatching(const VVI &w, VI &mr, VI &mc) {
  mr = VI(w.size(), -1);
  mc = VI(w[0].size(), -1);
  
  int ct = 0;
  for (int i = 0; i < w.size(); i++) {
    VI seen(w[0].size());
    if (FindMatch(i, w, mr, mc, seen)) ct++;
  }
  return ct;
}
\end{minted}


\begin{flushleft}
\textbf{\underline{FFT Algorithm}}
\end{flushleft}

\begin{minted}[autogobble,linenos=true,frame=single,linenos=false,breaklines]{c++}	
// Convolution using the fast Fourier transform (FFT).
//
// INPUT:
//     a[1...n]
//     b[1...m]
//
// OUTPUT:
//     c[1...n+m-1] such that c[k] = $sum_{i=0}^k a[i]b[k-i]$
//
// Alternatively, you can use the DFT() routine directly, which will
// zero-pad your input to the next largest power of 2 and compute the
// DFT or inverse DFT.
#include <iostream>
#include <vector>
#include <complex>

using namespace std;

typedef long double DOUBLE;
typedef complex<DOUBLE> COMPLEX;
typedef vector<DOUBLE> VD;
typedef vector<COMPLEX> VC;

struct FFT {
  VC A;
  int n, L;

  int ReverseBits(int k) {
    int ret = 0;
    for (int i = 0; i < L; i++) {
      ret = (ret << 1) | (k & 1);
      k >>= 1;
    }
    return ret;
  }

  void BitReverseCopy(VC a) {
    for (n = 1, L = 0; n < a.size(); n <<= 1, L++) ;
    A.resize(n);
    for (int k = 0; k < n; k++) 
      A[ReverseBits(k)] = a[k];
  }
  
  VC DFT(VC a, bool inverse) {
    BitReverseCopy(a);
    for (int s = 1; s <= L; s++) {
      int m = 1 << s;
      COMPLEX wm = exp(COMPLEX(0, 2.0 * M_PI / m));
      if (inverse) wm = COMPLEX(1, 0) / wm;
      for (int k = 0; k < n; k += m) {
	COMPLEX w = 1;
	for (int j = 0; j < m/2; j++) {
	  COMPLEX t = w * A[k + j + m/2];
	  COMPLEX u = A[k + j];
	  A[k + j] = u + t;
	  A[k + j + m/2] = u - t;
	  w = w * wm;
	}
      }
    }
    if (inverse) for (int i = 0; i < n; i++) A[i] /= n;
    return A;
  }

  // c[k] = sum_{i=0}^k a[i] b[k-i]
  VD Convolution(VD a, VD b) {
    int L = 1;
    while ((1 << L) < a.size()) L++;
    while ((1 << L) < b.size()) L++;
    int n = 1 << (L+1);

    VC aa, bb;
    for (size_t i = 0; i < n; i++) aa.push_back(i < a.size() ? COMPLEX(a[i], 0) : 0);
    for (size_t i = 0; i < n; i++) bb.push_back(i < b.size() ? COMPLEX(b[i], 0) : 0);
    
    VC AA = DFT(aa, false);
    VC BB = DFT(bb, false);
    VC CC;
    for (size_t i = 0; i < AA.size(); i++) CC.push_back(AA[i] * BB[i]);
    VC cc = DFT(CC, true);

    VD c;
    for (int i = 0; i < a.size() + b.size() - 1; i++) c.push_back(cc[i].real());
    return c;
  }

};

int main() {
  double a[] = {1, 3, 4, 5, 7};
  double b[] = {2, 4, 6};

  FFT fft;
  VD c = fft.Convolution(VD(a, a + 5), VD(b, b + 3));

  // expected output: 2 10 26 44 58 58 42
  for (int i = 0; i < c.size(); i++) cerr << c[i] << " ";
  cerr << endl;
  
  return 0;
}
\end{minted}

\begin{flushleft}
\textbf{\underline{Segment Tree With 2 different Lazy Propogation}}\\
\textbf{\underline{Tested on Spoj http://www.spoj.com/problems/SEGSQRSS/}}\\
\end{flushleft}

\begin{minted}[autogobble,linenos=true, frame=single,breaklines]{c++}
const int N = (int)1e5 + 10; 
int a[N], n , q; 
/*Reset L[i] to lazy(), t[i] = node()*/
struct node{
	LL sum1, sum2; 
	node() : sum1(0), sum2(0){}
	node(LL temp, LL _temp) : sum1(temp), sum2(_temp){}
}t[N << 2]; 

struct lazy{
	int t1, t2, x1, x2; // represents the type of the query we want to make. 
	lazy() : t1(-1) , t2(-1), x1(0), x2(0){}
	lazy(int a, int b, int c , int d): t1(a), t2(b), x1(c), x2(d){}
}L[N << 2]; 

node combine(node & a, node & b){
	node c;
	c.sum1 = a.sum1 + b.sum1; 
	c.sum2 = a.sum2 + b.sum2; 
	return c; 
}

void propogate1(int x,int l , int r){
	lazy & la = L[x]; 
	if(la.t1 != -1){
		int & num = la.x1; 
		t[x].sum1 += (r - l + 1)*(1LL)*(num*num) + 2*num*1LL*(t[x].sum2); 
		t[x].sum2 += (r - l + 1)*1LL*(num);
		if(l != r){
			L[x << 1].t1 = true; L[x << 1].x1 += num; 
			L[x << 1|1].t1 = true; L[x << 1|1].x1 += num; 
		}
		num = 0; 
		la.t1 = -1; 
	}
}

void propogate2(int x,int l , int r){
	lazy & la = L[x]; 
	if(la.t2 != -1){
		int & num = la.x2; 
		t[x].sum1 = (r - l + 1)*1LL*(num*num); 
		t[x].sum2 = (r - l + 1)*1LL*num; 
		if(l != r){
			L[x << 1].t1 = -1;L[x << 1].x2 = num; L[x << 1].t2 = true; L[x << 1].x1 = 0;  
			L[x << 1|1].t1 = -1;L[x << 1 | 1].x2 = num; L[x << 1 | 1].t2 = true; L[x << 1 | 1].x1 = 0; 
		}
		num = 0; 
		la.t2 = -1; 
	}
}

void update2(int x, int start , int end, int l , int r , int num){
	propogate2(x, start, end); 
	propogate1(x , start, end); 

	if(start > end || end < l || start > r)
		return; 

	if(start >= l && end <= r){ 
		L[x].t2 = true; L[x].x2 = num;
		propogate2(x, start, end);  
		return; 
	}

	int mid = (start + end) >> 1; 
	update2(x << 1, start, mid, l , r, num); update2(x << 1 | 1 , mid + 1, end, l , r , num); 
	t[x] = combine(t[x << 1], t[ x << 1 | 1]); 
}

void update1(int x, int start , int end, int l , int r , int num){
	propogate2(x , start, end); 
	propogate1(x, start, end) ; 

	if(start > end || end < l || start > r)
		return; 

	if(start >= l && end <= r){ 
		L[x].t1 = true; L[x].x1 += num; 
		propogate1(x, start, end); 
		return; 
	}

	int mid = (start + end) >> 1; 
	update1(x << 1, start, mid, l , r, num); update1(x << 1 | 1 , mid + 1, end, l , r , num); 
	t[x] = combine(t[x << 1], t[ x << 1 | 1]); 
}

LL query(int x, int start, int end, int l , int r){
	propogate2(x, start, end); 
	propogate1(x, start, end); 
	if(start > end || start > r || end < l)
		return 0; 

	if(start >= l && end <= r)
		return t[x].sum1; 

	int mid = (start + end) >> 1; 
	return query(x << 1 , start, mid , l , r) + query(x << 1 | 1 , mid + 1, end, l , r); 
}

void build(int x, int l , int r){
	if(l == r){
		t[x].sum1 = (a[l]*a[l]); 
		t[x].sum2 = a[l]; 
		return; 
	}
	int mid = ( l + r ) >> 1; 
	build(x << 1,l, mid); build(x << 1 | 1, mid + 1, r); 
	t[x] = combine(t[x << 1] , t[x << 1 | 1]); 
	return; 
}
\end{minted}


\begin{flushleft}
\textbf{\underline{Sublime Settings}}\\
Build System
\end{flushleft}

\begin{minted}[autogobble,linenos=true,frame=single,breaklines=true]{python}
{
	"cmd": ["g++ -std=c++11 ${file} -o ${file_path}/${file_base_name} && ${file_path}/${file_base_name}</home/equinox/Documents/CP/inputf.in>	
	/home/equinox/Documents/CP/outputf.in"],
	"shell":true
}
\end{minted}

\begin{flushleft}
\textbf{Snippet Sublime Settings}
\end{flushleft}

\begin{minted}[autogobble,linenos=true,frame=single,breaklines=true]{c++}
<snippet><content><![CDATA[
#define _CRT_SECURE_NO_WARNINGS
#include <bits/stdc++.h>
using namespace std;
using LL = long long; 
typedef long long               ll;
typedef pair<int, int>          pii;
typedef vector<int>             vi;
#define SYNC		ios_base::sync_with_stdio(0);cin.tie(0); 
ll MOD = 1000000007;
#define rep(i,b)   for (int i=0; i < b; i++)
#define REP(i,n) 	for(int i = 0; i < n;i++)
#define REPN(i,n) 	for(int i = 1; i <= n;i++)
#define ALL(x)  	 (x).begin, (x).end()
#define fi           first
#define se           second
#define pp           push_back
#define pb 		     emplace_back
#define dzx 		cerr<<"here";
#define deb(x)		cerr << #x << " here "<< x;
#define debn(x)		cerr << #x << " here " << x << "\n"; 
//START
int main()
{
	SYNC
	
	return 0;
}
]]></content>
	<!-- Optional: Set a tabTrigger to define how to trigger the snippet -->
	 <tabTrigger>zxc</tabTrigger> 
	<!-- Optional: Set a scope to limit where the snippet will trigger -->
	<scope>source.c++</scope> 
</snippet>
\end{minted}


\begin{flushleft}
\textbf{\textit{Tree Algorithms}}\\
\textbf{\underline{Centroid, Centre and Diameter of a tree}}\\
\end{flushleft}

\begin{minted}[autogobble,linenos=true,breaklines, frame=single, tabsize=0.5]{c++}
const int N = (int)1e5 + 10;
vector<vector<int>> centroid_tree, g;
bool CentMark[N]; int child[N];  

void dfs(int u,int p, bool cond, int & centroid, int nodes){
	child[u] += cond;
	bool possible = true;  	
	for(int i : g[u]){
		if(i != p && !CentMark[i]){
			dfs(i,u, cond, centroid, nodes);
			child[u] += (cond ? child[i] : 0); 
			possible &= (!cond ?((child[i]) <= (nodes/2)): false);
		}
	}
	if(possible && (nodes - child[u]) <= (nodes)/2)
		centroid = u; 
	return;
}

int find_centroid(int src){
	memset(child, 0 , sizeof child);
	int centroid = -1;
	dfs(src, -1, 1, centroid, 0); 
	dfs(src,-1,0, centroid, child[src]);

	CentMark[centroid] = true; 
	return centroid; 
}

int decompose(int u){
	int centroid = find_centroid(u); 

	for(int i : g[centroid]){
		if(!CentMark[i]){
			int SubTreeCentroid = decompose(i); 
			centroid_tree[SubTreeCentroid].push_back(centroid); 
			centroid_tree[centroid].push_back(SubTreeCentroid);
		}
	}

	return centroid; 
}
\end{minted}


\begin{flushleft}
\underline{\textbf{Centre and Diameter of Tree}}\\
To find the centre once we got the Centre is given by simple formula:-\\
\textbf{$diam(G) = 2maxlevel + |C| - 1$}\\
Below is the implementation for the same:-
\end{flushleft}

\begin{minted}[autogobble,linenos=true,frame=single,breaklines]{c++}
const int N = (int)1e5 + 10; // Number of nodes in graph
int maxlevel = 0; // Level of center of graph
int level[N]; // Level of node
int degree[N]; // Degree of node
vector<vector<int>> g;

// Start from leaves
set<int> center(){
    set <int> c; // Center of graph
    queue <int> q; // Queue for algo
    
    for (int i = 0; i < N; i++) {
        if (degree[i] == 1) {
            q.push(i);
        }
    }

    while (!q.empty()) {
        int v = q.front();
        q.pop();

        // Remove leaf and try to add its parent
        for (int i : g[u]) {
            degree[i]--;
            if (degree[i] == 1) {
                q.push(i);
                level[i] = level[v] + 1;
                maxlevel = max(maxlevel, level[i]);
            }
        }
    }

    for (int i = 0; i < N; i++) {
        if (level[i] == maxlevel) {
            c.insert(i);
        }
    }
    return c; 
}
\end{minted}


%\begin{flushleft}
%Some Problems on DP on tree(My AC Solutions)\\
%\textbf{\textit{Given tree T with n nodes, how many subtrees (T') of T have at most K edges connected to (T - T')?}}
%{\underline{HackerRank Cut Tree(My AC link https://www.hackerrank.com/challenges/cuttree/submissions/code/33276651)}}
%\end{flushleft}

\begin{minted}[autogobble,linenos=true,breaklines,frame=single]{c++}
const int N = 60; 
int n, k;
vector<vll> dp1; vector<LL> dp2; 
vector<vi> g;

void dfs(int u, int par) {
	
	int j = 0;

	vector<vll> dp3; 
	dp3.clear(); 
	dp3.assign(1LL * n + 2LL, vll(n*1LL + 1LL, 0LL));

	for (int i : g[u]) {
		if (par != i) {
			dfs(i, u);
			if (!j) {
				dp3[j][0] = dp3[j][1] = 1; 
				for (int K = 1; K <= k; K++) {
					if (K)
						dp2[u] += dp1[i][K - 1];
					dp3[j][K] += (dp1[i][K]);
				}
			}
			else {
				
				for (int K = 0; K <= k; K++) {
					if (K)
						dp2[u] += dp1[i][K - 1];
					if (K) {
						dp3[j][K] += dp3[j-1][K - 1]; 
					}
					for (int x = 0; x <= K; x++) {
						dp3[j][K] += (dp3[j - 1][K - x])*(dp1[i][x]);
					}
				}
			}
			dp2[u] += (dp2[i]);
			j++; 
		}
	}
	 
	if(j)
		dp1[u] = dp3[j - 1]; 

	if (!j) {
		dp1[u][0] = 1; 
		dp2[u] = 0; 
	}

}
int main() {
    
    //freopen("input.txt" , "r" , stdin); 
	cin >> n >> k; 

	g.clear(); g.resize(n + 1);
	
	dp1.assign(n + 2, vll(n*1LL + 1, 0LL)); 
	dp2.assign(1LL * n + 2LL, 0LL);

	for (int i = 0; i < n - 1; i++) {
		int a, b; 
		cin >> a >> b; 
		a--, b--; 
		g[a].push_back(b); g[b].push_back(a); 
	}
	dfs(0, -1); 
	LL ans = dp2[0]; 

	for (int i = 0; i <= k; i++) {
        ans += dp1[0][i]; 
	}

	cout << ans + 1 << endl; 
	return 0; 
}
\end{minted}

\begin{flushleft}
%\textbf{\underline{Model Solution for Persistent Segment Tree, RMQ, Suffix Arrays and LCP Arrays}}\\
Tested on CodeChef QString\\
\end{flushleft}

\begin{minted}[autogobble,linenos=true,breaklines,frame=single]{c++}
#include <bits/stdc++.h>
 
using namespace std;
 
typedef long long LL;
 
// This class takes as an argument an array of size N. 0-based index for the given array. 
// Example : A[] = {1,2,3,4,6}, call the constructor by (A, 5)
// It assumes that given elements will be unique and lie in the range [1, N].
// Method KthMinimum(l, r, k) return the kth minimum number in the given sub-array [l...r] in O(log N) time. l, r, k all are 1-based.
// Method RankingOfK(l, r, k) return the rank of k in the sub-array [l...r] in O(log N) time. l, r, k are all 1-based.
// This class does not cares about the validity of the query. It assumes that given query is valid. So be careful while using it.
class PersistentSegmentTree{
public:
	PersistentSegmentTree(int A[], int n){
		N = n;
		head = (node **)calloc(N+1, sizeof(node *));
		head[0] = create(1, N);
		for(int i = 0; i<N; ++i){
			head[i+1] = insert(1, N, A[i], head[i]);
		}
	}
	int KthMinimum(int l, int r, int k){
		return KthMinimum(head[l-1], head[r], k, 1, N);
	}
	int RankingOfK(int l, int r, int k){
		return RankingOfK(head[l-1], head[r], k, 1, N);
	}
	~PersistentSegmentTree(){
		free(head);
	}
private:
	struct node{
		int val;
		node * left;
		node * right;
	};
	int N;
	node ** head;
	node * make_node(int val){
		node * nd = (node *)calloc(1, sizeof(node));
		nd -> val = val;
		nd -> left = NULL;
		nd -> right = NULL;
		return nd;
	}
	node * create(int l, int r){
		node * nd = make_node(0);
		int m = (l+r) >> 1;
		if(l != r){
			nd -> left = create(l, m);
			nd -> right = create(m+1,r);
		}
		return nd;
	}
	node * insert(int l, int r, int v, node * n){
		node * nd = make_node(0);
		if(l == r){
			assert(l == v);
			nd -> val += 1;
			return nd;
		}
		int m = (l+r) >> 1;
		if(v <= m) nd -> left = insert(l, m, v, n->left), nd->right = n->right;
		else nd -> right = insert(m+1, r, v, n->right), nd->left = n->left;
		(nd -> val) = (nd -> left) -> val + (nd -> right) -> val;
		return nd;
	}
	int KthMinimum(node * l, node * r, int k, int i, int j){
		if(l->left == NULL and l->right == NULL){
			assert(r->left == r->right and r->left == NULL);
			assert(k==1 and i==j);
			return i;
		}
		int m = (i+j)>>1;
		int val = (r->left->val) - (l->left->val);
		if(val >= k) return KthMinimum(l->left, r->left, k, i, m);
		else return KthMinimum(l->right, r->right, k-val, m+1, j);
	}
	int RankingOfK(node * l, node * r, int k, int i, int j){
		if(l->left == NULL and l->right == NULL){
			assert(r->left == r->right and r->left == NULL);
			assert(k==i and i==j);
			return 1;
		}
		int m = (i+j)>>1;
		int val = (r->left->val) - (l->left->val);
		if(k <= m) return RankingOfK(l->left, r->left, k, i, m);
		else return RankingOfK(l->right, r->right, k, m+1, j) + val;
	}};
// This class takes as an argument an array of size N. 0-based index for the given array. 
// Example : A[] = {1,2,3,4,6}, call the constructor by (A, 5)
// Method findMin(l, r) return the minimum element in the range [l...r] in O(1) time after preprocessing time of O(N log N).
// l and r are 1-based index. Memory used is O(NlogN) as well.
// this class doesn't checks the validity of the inputs. Be careful why using this class
class RMQ
{
public:
	RMQ(int A[], int n){
		RMQ::init();
		H = ceil(log2(n));
		N = n;
		M.resize(N+1);
		for(int i=0;i<=N;++i)
			M[i].resize(H+1);
		for(int i=0;i<=H;++i){
			int jump = ((1<<i)>>1);
			for(int j=1;j<=N;++j)
			{
				if(i == 0) M[j][i] = A[j-1];
				else if(j+jump <= N) M[j][i] = min(M[j][i-1], M[j+jump][i-1]);
				else  M[j][i] = M[j][i-1];
			}
		}
	}
	~RMQ(){
		for(int i=0;i<=N;++i) M[i].clear();
		M.clear();
	}
	int findMin(int l, int r){
		if(l>r) return 0;
		int d = A[r-l+1];
		r = r - (1<<d) + 1;
		return min(M[l][d], M[r][d]);}
private:
	int H, N;
	vector<vector<int> > M;
	int A[2000001];
	void init(){
		memset(A, -1, sizeof A);
		for(int i=0; (1<<i)<=2000000; ++i){
			A[(1<<i)] = i;
		}
		for(int i=1;i<=2000000;++i){
			if(A[i] != -1) continue;
			A[i] = A[i-1];
			}return;}};
\end{minted}

\begin{minted}[frame=single,autogobble,linenos=true,linenos=true,breaklines]{c++}		
template<typename T> T gcd(T a , T b){return (a ? gcd(b % a , a): b);} //supposing a is small and b is large.
template<typename T> pair<T,T> extend_euclid(T a, T b){ //supposing a is small and b is large.
	pair<T,T> a_one = {1, 0} , b_one = {0 , 1}; 
	if(!b)return a_one;
	while(a){
		T q = b / a; T r = b % a;
		T dx = b_one.first - q*a_one.first; 
		T dy = b_one.second - q*a_one.second; 
		b = a; a = r;
		b_one = a_one; 
		a_one = {dx , dy}; 
	}
	return b_one;   
}

int main(){
	LL a, m; cin >> a >> m;
	auto ans = extend_euclid(a, m); 
	LL x = (ans.first + m)%m; //Inverse Modulo (m) $ ax=1 mod(m) and gcd(a,m) == 1
	cout << (ans.first + m) % m << endl; 
	return 0; 
}
\end{minted}
\end{multicols}
\end{landscape}
\end{document}




















